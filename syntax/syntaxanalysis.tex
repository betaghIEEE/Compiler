\documentclass[11pt]{article}
\usepackage{graphicx}
\usepackage{amssymb}
\usepackage{epstopdf}
\DeclareGraphicsRule{.tif}{png}{.png}{`convert #1 `dirname #1`/`basename #1 .tif`.png}

\textwidth = 6.5 in
\textheight = 9 in
\oddsidemargin = 0.0 in
\evensidemargin = 0.0 in
\topmargin = 0.0 in
\headheight = 0.0 in
\headsep = 0.0 in
\parskip = 0.2in
\parindent = 0.0in

\newtheorem{theorem}{Theorem}
\newtheorem{corollary}[theorem]{Corollary}
\newtheorem{definition}{Definition}

\title{Notes on Syntax Analysis}
\author{Daniel Beatty}
\begin{document}
\maketitle


\section {Context-Free Grammars}
``Many programming language constructs have an inherently recursive structure that can be defined by context-free grammars.''    One construct is:
\[ S \rightarrow \textbf{if } E \textbf { then } S_1 \textbf{ else } S_2 \]  

\section {Top down parsing}
The purpose of this section is to introduce the basic ideas behind top-down parsing and show how to construct an efficient non-backtracking form of top-down parser called a predictive parser.   In the process, we construct a parse tree for an input string from the root and create the nodes of the parse tree in pre-order.  

There are some catch phrases used through out this section, predictive parsing and recursive descent.  Also, the term backtracking is used significantly.  It is said that a top-down LL(1) parser seldom needs backtracking for a context free grammar, and tends to be inefficient on context sensitive grammars.  

A left-recursive grammar can cause a a recursive descent parser, even one with backtracking to go into an infinite loop.  

Predictive Parser Qualities:
\begin{itemize}
\item Eliminates left recursion 
\item left factors
\item Proper alternatives are detected by examining the first symbols derives.
\end{itemize}


Transition Diagrams for Predictive Parsers 

Differences between lexical analyzers and predictive parser transition.  
\begin{itemize}
\item One diagram for each non-terminal 
\item Labels are both tokens and non-terminals
\item Transition on non-terminal equates to a procedure call
\end{itemize}

To apply a transition diagram:
\begin{itemize}
\item Apply left recursion removal
\item Left factor the grammar
\item Create an initial and final return state
\item For each production $A\to X_1, X_2, ...,X_n$ create a path from the initial to the final state with edges labeled $X_1, X_2, ...,X_n$.
\end{itemize}

What happens for a predictive parser which works off the transition diagram?
\begin{enumerate}
\item It begins in the start state for the start symbol.
\item After some actions, the parser arrives in state $s$.
\begin{itemize}
\item Possibility: There is an edge to $t$ labeled a if the next input symbol is $a$, then the transition to $t$ occurs.  
\item There is an edge to $t$ labeled by non-terminal $A$.\begin{itemize}
\item The parser goes to the start state of $A$ without moving the input cursor.
\item If a final state in $A$ is reached, the transition is made.
\end{itemize}
\item There is an edge to $t$ labeled by $\epsilon$ .
\begin{itemize}
\item Transition is made without advancing the input cursor.
\end{itemize}


\end{itemize}

\end{enumerate}
``A predictive parsing program based on a transition diagram attempts to match terminal symbols against the input, and makes a recursive procedure call whenever it has to follow an edge labeled by a nonterminal.''
\begin{itemize}
\item For nondeterminism this does not work
\item For determinism this works
\end{itemize}

\textbf{Non-recursive predictive parsing}
A stack may be used explicitly to build a non-recursive predictive parser.  Problems exist in determining the production to be applied.  Solutions to these problems include parse table look up.   Table-driven predictive parser (reference the diagram, page 184 ``A table driven predictive parser.'').




\section {LL(1)}
$LL(1)$ is a left to right scan of input Leftmost de

Function of boolean 

All functions are boolean.  

Example: $A\to aBa |C |\epsilon$ this a generic case rule.  There is function that implements this rule.  There is a selection set associated with $A$ ($SS(A)$).  Assumption ??? 
function A: boolean 
is this syntax rule satisfied.  
If next token equals 'a'  (if nt = 'a' ) then do the following:
	get (nt)
	consume (a)
	if (B) then if nt = 'a' then consume (a) A := true
		else 
		error A:= false.
	else
		A:= false 
else 
	if C then A:= true
else 
	if $nt \in SS(A)$ then A := true
	else 
		error
		A:=false
		
Note that all consumes call the lexical analyzer.  

Note about if-else structures.  In cases of if a if b if c is equivalent to an and, and if else if is analogous to an or.   Reference the book on first, follow and recursive decent parsing.  

Compiler-compilers came out of bottom-up compilers.  Donald Knuth made this contribution to comp sci.  Optimizations comes out of bottom-up compilers.  Produce the selection sets and there is a full precise specification of the grammar.  


Example grammar:

2:30:35

reference 2:32 for E1  

next page 2:37:35

function T1: boolean reference 2:40 

Tracing of the example function:

example finished at 3:11.  

Knuth and   argument.  

There is a viable prefix property.  Parses until the first error.  

\section {September 28}
note at 2:10

Selection monitor  at 2:13


$C\to E | E | <$

note 2:24

Palindrome problem 2:34

Slightly altered grammar 2:37

Solution 2:45

Delimiter to the palindrome problem is the semi-colon (;).

What the expression looks like: 2:50


Analyze language generated from E, 2:55

Trouble with comma's 3:02

Structured stand point 3:06


Error recovery and begin semantic analysis.  Syntax analysis should be done by the end of the weekend.  


 \bibliography{logic}
\bibliographystyle{abbrv}



 \end{document} 
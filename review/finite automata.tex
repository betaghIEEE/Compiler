 \section{Finite  Automata}
`` A \textsl{recognizer} for a language is a program that takes as input a string $x$ and answers `yes' if $x$ is a sentence of the language and `no' otherwise.''  Non-deterministic finite automata has more than one transition out of a state on the same symbol.   A deterministic finite automata has only one state transition per symbol.  

\textbf {\Large Non-deterministic Finite Automata}
\begin{quote}
A non-deterministic finite automata (NFA) is a mathematical model that consists of 
\begin{enumerate}
\item a set of states S
\item a set of input symbol $\Sigma$ ( the input symbol alphabet)
\item a transition function move that maps state-symbol pairs to sets of states
\item a state $s_0$ that is distinguished as the start (or initial) state 
\item a set of states F distinguished as accepting (or final) states
\end{enumerate}
\end{quote}
\cite {ullman}
``An NFA accepts an input string $x$ if and only if there is some path in the transition graph from the start  state to some accepting state, such that the edge labels along this path spell out $x$. ''

NFA is typically represent pictorially by a transition graph.  A transition graph is a graph where the states are the nodes and labeled edges represent the transition functions.  NFA may include edges represented by $\epsilon$.  

A NFA may also be represented by a transition table.  A transition table consists of:
\begin{itemize}
\item a row for each state
\item a column for each symbol
\item An entry in each cell representing the states that can be reached by the row's state and column's input symbol
\end{itemize}

\textbf {\Large Deterministic Finite Automata}
\begin{quote}
A deterministic finite automaton (DFA) is a special case of NFA in which 
\begin{enumerate}
\item no state has an $\epsilon$-transition
\item for each state $s$ and input symbol $a$, there is at most one edge labeled $a$ leaving $s$
\end{enumerate}
\end{quote}

\textbf{\Large Example: Minimal DFA}  $NFA\to DFA$ $\epsilon$ closure / composite states

``The DFA uses its state to keep track of all possible states the NFA can be in after reading each input symbol.''  

\textbf {Algorithm} Subset construction: Constructing a DFA from an NFA

\textsl{Input} An NFA $N$

\textsl{Output:} A DFA $D$ accepting the same language

\textsl{Goal:} ``Each DFA state is a set of NFA states and we construct $D$-tran so that $D$ will simulate in parallel all possible moves N can make on a given input string.''  

\textsl{Method.} Construct a transition table $D$-tran for $D$.  Apply $\epsilon$ closure and move methods, 
\begin{table}[htdp]
\caption{default}
\begin{center}
\begin{tabular}{|c|c|}
$\epsilon$-closure(s) & Set of NFA states reachable from NFA state $s$ on $\epsilon$-transitions alone \\
$\epsilon$ -closure(T) & Set of NFA states reachable from some NFA state $s\in T$ on $\epsilon$-transitions alone \\
$move(T,a)$ & Set of NFA states to which there is a transition on input symbol $a$ from some NFA state $s\inT$
\end{tabular}
\end{center}
\label{default}
\end{table}%

{\large Notes on subset construction}
\begin{enumerate}
\item $\epsilon$-closure ($s_0$) shows a pseudo-state composed of the states $s\in T$ reachable from $s_0$.  For symbol $a$, these states include $\epsilon closure(move(s_0,a))$.
\item
Each state of $D$ corresponds to a set of NFA states that $N$ could be after reading a sequence of of input symbols including all possible $\epsilon$-transitions before or after the symbols are read.  
\item 
Starting state of $D$ is $\epsilon closure (s_0)$.  
\item
An accepting state in $D$ is defined as the state  is a set of NFA states containing at least one accepting state of $N$.  
\item
A simple algorithm to compute $\epsilon$ - closure (T) uses a stack to hold states whose states whose edges have not been checked for $\epsilon$ labeled transitions.  
\end{enumerate}


\section {Example: Lexical Analysis Through Semantics}

Note that whole point of the lexical analyzer in this case is to identify tokens.   The calling program reads in a buffer of characters (lines of 80 characters or less).  This string is passed to the lexical analyzer.  As a debugging tool, each string will be printed when processed.  Also comments as to success of failure shall also be printed.  

It has been alluded that the syntax analyzer will actually be doing the calling, trying to acquire tokens.  On feature that appears to be necessary is a common symbol table.  Whether this symbol table is global, is simply referenced by both is something to be tested.   There is a suggestion that makes sense.  Pre-load the symbol table with all reserved words and symbols.  All reserved words have a token type identical to its index in the symbol table.  The suggested record structure is as follows:
\begin{itemize}
\item Name string
\item Location 
\item Token Type 
\item Index of Token
\item Constants
\item Integer
\end{itemize}
\textbf {\Large Special Suggestions}
\begin{enumerate}
\item Operators, special symbols, and spaces are delimiters.  How the delimiter is determined is immaterial.  What matters is that when one of these delimiters is encountered, the token acquisition stops.  
\item Generate Selection Sets for the grammar;
\item You are to write a phased implementation of a compiler to translate a program written in the language described above into an object language to be defined at a later date.  Your compiler will consist of a lexical analyzer (or scanner), syntax analyzer, semantic analyzer, and code generator.
\end{enumerate}



Reserved Words/Symbols:
\begin{enumerate}
\item program
\item array
\item integer
\item read
\item write
\item rdln
\item wrln
\item when
\item until
\item from
\item from
\item ..
\item delimiter symbols
\begin{itemize}
\item eof
\item \left[
\item \}
\item [
\item ]
\item ;
\item ,
\item (
\item )
\item :=
\item $<$
\item $>$
\item =
\item $<=$
\item $>=$
\item $<>$
\item $+$
\item $-$
\item $*$
\item /
\item :
\end{itemize}
\item and
\item or
\item mod
\end{enumerate}


Syntax is case insensitive.  


\textbf{\Large Tokens of the Language}
Tokens of the language can be either a space ($S_0$), special symbol ($S_s$), identifier, numerical constant, or a reserved word.   Identifiers and reserved words can be considered to almost the same except, the reserved words are preloaded into the symbol table.  
\begin{itemize}
\item $S_d = S_s \bigcup S_0$
\item $u_c = \{A,...,Z\}$
\item $l_c = \{a, ..., z \}$
\item $n_0 = \{0,...,9\}$
\end{itemize}

\textbf{\Large A cute token analyzer} 
\\
\begin{itemize}
\item input: String s integer position
\item output
	Token identifier
\item Support variables: character c, integer tokenid, integer index
\end{itemize}



\begin{thebibliography}{99}
\bibitem {ullman} Jeffrey D. Ulman, Ravi Sethi, Alfred V. Aho \textsl{Compilers: Principles, Techniques and Tools} copywrite 1986 by Bell Telephone Laboratories, Incorporated
\end {thebibliography}
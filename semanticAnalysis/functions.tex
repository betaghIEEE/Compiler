\documentclass[11pt]{article}
\usepackage{geometry}                % See geometry.pdf to learn the layout options. There are lots.
\geometry{letterpaper}                   % ... or a4paper or a5paper or ... 
%\geometry{landscape}                % Activate for for rotated page geometry
%\usepackage[parfill]{parskip}    % Activate to begin paragraphs with an empty line rather than an indent
\usepackage{graphicx}
\usepackage{amssymb}
\usepackage{epstopdf}
\DeclareGraphicsRule{.tif}{png}{.png}{`convert #1 `dirname #1`/`basename #1 .tif`.png}

\title{Notes on Functions}
\author{Dan Beatty}
%\date{}                                           % Activate to display a given date or no date

\begin{document}
\maketitle
%\section{}
%\subsection{}



Functions in procedural languages have the following properties at definition
\begin{itemize}
\item Location of the code
\item Return value
\item Parameters (number and types of parameters).
\item Activation Records have
\begin{itemize}
\item Pass by reference or value (pass by value in this case)
\item Pass by value requires memory allocation at function activation time. 
\item the activation record
\item the call by value information 
\item  return location get placed on the stack
\item All local variable to the calling scope
\end{itemize}
\item Any calls to the function must include the activation record and information contained with in. 
\item Activation records are activated (pushed) and deactivated (popped).   
\end{itemize}

Functions cause the quad table to have 2 more entries.  
\begin{itemize}
\item where does it begin
\item Active status 
\end{itemize}

\newpage
Assumptions on functions
\begin{enumerate}
\item One function active at a time (disallowing a function calling another)
\item Nesting of function is not allowed either. 
\item All function variable references will be local to the function.  (no globals)
\item No local variables other than the parameters themselves.  
\end{enumerate}



 
 \begin{table}[htdp]
\caption{The production for a function will need to have the following}
\begin{center}
\begin{tabular}{l}
FUNC $\to$ function id (params) \{ S \} \\
\hline
PARAMS  $\to$  id P1 \\
 P1 $\to$ , id P1  | $\epsilon$ \\
\end{tabular}
\end{center}
\label{default}
\end{table}%


 \begin{table}[htdp]
\caption{In fix production handling function}
\begin{center}
\begin{tabular}{l}
F $\to$ id $|$ cons $|$ ( E )  $|$ func (args)  \\
S $\to $ $|$ func := E $|$   
\end{tabular}
\end{center}
\label{default}
\end{table}%

\textbf{Semantic Actions }
 \begin{table}[htdp]
\begin{center}
\begin{tabular}{l|ll}
A & function := id \\
\hline
B	&	i := 2 ; \\
	&	param := pop; \\
	&	while param $\neq$ \# do \\
	&	\{	\\
	&	\hspace{0.5cm}	ST [param, context ] := function ; \\
	&	\hspace{0.5cm}	i := i +1 ; \\
	&	\hspace{0.5cm}	param := pop \\
	& 	\}	\\
	&	ST [ function , \# params] := i - 2;  // Number of parameters  \\
	& 	ST [next , name] := function.ret ; \\
	&	next ++ ; 	\\
	&	i := 2;    // offset to be kept   \\
	&	n := function + function [ function \# params ] ; \\
	&	for j := function + 1 to n do \\
	&	\{ \\
	&	\hspace{0.5cm}	ST [j, loc] := i; \\
	&	\hspace{0.5cm}	i ++ ; \\
	&	\} \\
	&	ST [function , Loc ] := NQ; \\
	\hline
C	& 	push (function, ret).
	&	return := active\_ar $\hat{ }$ .ar[1] ; \\
	&	genquads ( pop, function, \_ , \_ ) & manage ar stack \\
	&								& and if ar stack \\
	&								& is empty assign zero  - active function \\
\end{tabular}
\end{center}
\label{default}
\end{table}%


push
transfer args
compare return address
jump to quads to and from 



\begin{itemize}
\item Header for active  and activation record
\item which function ST
\end{itemize}

Active 
\begin{itemize}
\item ST entry for the active function 
\item Header to the AR stack 
\begin{table}[htdp]
%\caption{default} 
\begin{center}
\begin{tabular}{ll|l}
A & genquad ( := , push (id) , \_ , active )  & active function , active ar (2 items)\\
	&	n := st [id, \# of params ] ;	\\
	&	i := 2	\\
	&	for j := n downto i do \\
	& \{ \\
	& \hspace{05.cm} active\_ar$\hat{ } $ . ar[j] := pop ; \\
	& \} \\
	& empty := pop ; \\
	& genquad (jmp, \_, \_ , st [id, loc] ) ; \\
	& active ar $\hat{ } $ . ar[1] := NQ;  \\

\end{tabular}
\end{center}
\label{default}
\end{table}%

\end{itemize}


\begin{table}[htdp]
%\caption{default}
\begin{center}
\begin{tabular}{|ll}
if active.func = -1 then  \\
\hspace {0.5cm} push (id) ;  \\
else
\hspace {0.5cm} push ( active\_ar [ST[id, loc] ) ; \\
\end{tabular}
\end{center}
\label{default}
\end{table}%

Concept of a ``Display'' is for nested procedures and functions.   


procedure x

	var a, b, c;
	
	procedure y
		var d, a
			y call itself
	end;
	y;
end;

\end{document}  



\documentclass[11pt]{article}
\usepackage{graphicx}
\usepackage{amssymb}
\usepackage{epstopdf}
\DeclareGraphicsRule{.tif}{png}{.png}{`convert #1 `dirname #1`/`basename #1 .tif`.png}

\textwidth = 6.5 in
\textheight = 9 in
\oddsidemargin = 0.0 in
\evensidemargin = 0.0 in
\topmargin = 0.0 in
\headheight = 0.0 in
\headsep = 0.0 in
\parskip = 0.2in
\parindent = 0.0in

\newtheorem{theorem}{Theorem}
\newtheorem{corollary}[theorem]{Corollary}
\newtheorem{definition}{Definition}

\title{Parsers: A set notes from Dan Cooke's Manual and Alfred Aho, Ravi Sethi, and Jeffrey Ulman's Compilers Principles, Techniques and Tools}
\author{Daniel Beatty}
\begin{document}
\maketitle


 The Role of the Parser
 ``The parser obtains a string of tokens from the lexical analyzer, and verifies that the string can be generated by the grammar for the source language. ''  It is expected that the parser should recover from commonly occuring errors so that it can continue processing the remainder of its input.   
 
 The parser is typically applied to context free grammars.  
 
 ``Why use regular expressions to define the lexical syntax of a language?''
\begin{enumerate}
\item The lexical rules of a language are frequently quite simple, and to describe them we do not need a notation as powerful as grammars
\item Regular expression generally provide a more concise and easier to understand notation for tokens than grammars.
\item More efficient lexical analyzers can be constructed automatically from regular expressions than from arbitrary grammars.
\item Separating the syntactic structure of a language into lexical and non-lexical parts provides a convenient way of modularizing the front end of a compiler into two manageable-sized components.  
\item Regular expressions are most useful for describing the structure of lexical constructs such as identifiers, constants, keywords, and so forth.  
\item Grammars are useful in describing nested structures such as balanced parentheses, matching begin and end statements, corresponding if-then-else statements, and the like.
\end{enumerate}

\newpage
Errors are expected to be reported if they occur lexically, syntactically, semantically, or logically.  Three types of parsers are common for compilers.
The first compiler type is for universal parsing as done by the Cocke-Younger-Kasami algorithm.  Universal implies that it will compile any grammar; however, it is inefficient for commercial compilers.    Top-down, or bottom up compilers handle LL and LR grammars which is good enough in most cases.  However, not all grammars are LL or LR.  
An error handler has a simple to state set of goals:
\begin{itemize}
\item Should report the presence of errors clearly and accurately.
\item It should recover from errors quickly enough to detect subsequent errors.   
\item  Should not slow down the processing of correct programs.  
\end{itemize}
 LL and LR grammar(s) may allow the compiler to detect immediately an error produced by a incorrect program.  
 
 One common mistake in compiler design is the avalanche of spurious errors.  This is an inadequate job of recovery.  
 
 Error Recovery strategies include:
 \begin{enumerate}
\item Panic mode 
\item Phrase mode
\item Error Productions
\item Global Corrections
\end{enumerate}
Page 164


\textbf {\Large Elimination of Left Recursion}
Eliminating Left Recursion is one of the steps for preparing a CFG to be parsed.  The general rule is for mapping  $A\to A\alpha | \beta$ to a non-left recursive production
\begin{itemize}
\item $A \to \beta A'$
\item $A' \to \alpha A' | \epsilon $
\end{itemize}

Algorithm for eliminating left recursion:


\textbf {\Large Left Factoring}
Why is left factoring useful as a grammar transformation?  When it is not clear which of two alternative productions to use to expand a nonterminal $A$, we may be able to re-wrtie the $A$-productions to defer the decision until we have seen enough of the input to make the right choice.  Left-factoring is also known as removing common prefixes.  

Given a production of the form $A\to \alpha \beta _1 | \alpha \beta _2 | \gamma$

Transformed to:
\begin{itemize}
\item $A \to \alpha A' | \gamma$
\item $A' \to \beta _1 | \beta _2$
\end{itemize}

\textbf{Algorithm for Left Factoring a grammar:} \\

Input: Grammar G

Output: An equivalent left-factored grammar.

Method: For each non-terminal $A$ find the longest prefix $\alpha$ common to two or more of its alternatives.  If $\alpha \neq \epsilon$ i.e., there is a nontrivial common prefix, replace all the $A$ productions $A\to \alpha \beta _1 | \alpha \beta _2 | \gamma$ where $\gamma$ represents all alternatives that do not begin with $\alpha$ by 
\begin{itemize}
\item $A \to \alpha A' | \gamma$
\item $A' \to \beta _1 | \beta _2$
\end{itemize}

Here $A'$ is a new non-terminal.  Repeatedly apply this transformation until no two alternatives for a non-terminal for a non-terminal have a common prefix.  


\section \textbf {\Large Top Down Parsing Notes}
\textbf{Purpose:} To introduce the basic ideas behind top down parsing and show how to construct and efficient non-backtracking efficient non-backtracking form of top-down parser called a predictive parser.  

\textbf {Goal:} Construct a parse tree for an input string from the root and create the nodes of the parse tree in pre-order. 

\textbf {Buzz words} 
\begin{itemize}
\item Predictive parsing 
\item Recursive descent
\end{itemize}

\textbf {Backtracking}
\begin{itemize}
\item Seldom necessary for context free grammar.
\item inefficient on CSG
\end{itemize}

A left-recursive grammar can cause a recursive-descent parser, even one with back-tracking, to go into an infinite loop.  

Predictive Parser Qualities:
\begin{itemize}
\item Eliminates left recursion
\item left factors 
\item Proper alternatives are detected by examining the first symbols derives.
\end{itemize}

\textbf {Transition Diagrams for Parsers}

\textbf {Differences between lexical analyzer and predictive parser transition}
\begin{itemize}
\item One diagram for each non-terminal 
\item Labels are both tokens and non-terminals
\item Transition on tokens next symbols 
\item Transition on non-terminal equates to a procedure calls
\end{itemize}


\textbf{\Large To build a syntax analyzer}
\begin{enumerate}
\item Apply left recursion removal
\item Left factor the grammar 
\item Create initial and final (return) state
\item For each production, $A\to X_1, X_2, ..., X_n$ create a path from the initial to the final state with edges labeled $X_1, X_2, ..., X_n$.  
\end{enumerate}

\textbf{ \Large What happens for a predictive parser which works off the transition diagram?}
\begin{enumerate}
\item It begins in the start state for the start symbol.
\item After some actions, the parser arrives in state $S$.
\begin{itemize}
\item Possibility: There is an edge to $t$ labeled $a$.  If the next input symbol is $a$, then the transition to $t$ occurs.  
\item Possibility: There is an edge to $t$ labeled by non-terminal $A$
\begin{itemize}
\item The parser goes to the start state of $A$ without moving the input cursor.
\item If a final state in $A$ is reached, the transition is made
\end{itemize}
\item Possibility: There is an edge to $t$ labeled by $\epsilon$.  Transition is made without advancing the input cursor. 
\end{itemize}

\end{enumerate}


reference: page 184
\textbf {Big idea} ``A predictive parsing program based on a transition diagram attempts to match terminal symbols against the input, and makes a potentially recursive procedure call, whenever it has to follow an edge labeled by a nonterminal.''
\begin{itemize}
\item For non-determinism this does not work
\item For deterministic automata this can work.
\end{itemize}

\textbf {Non-recursive Predictive Parsing}
A stack may be used explicitly to build a non-recursive predictive parser.
\begin{itemize}
\item Problems determining the production to be applied 
\item Solution: parse table look up- table driven predictive parser.
\end{itemize}

\begin{quote}
\textbf {First, Follow, and Selection Sets}
\begin{itemize}
\item First and follow sets form functions associated with the grammar and aids in the construction of a predictive parser by filling in the predictive parsing table.  
\item The FIRST($\alpha$) is the set of terminals that begin the set of strings derived from $\alpha$ 
\item The FOLLOW($A$) for a non-terminal $A$, is the set of terminals that can appear immediately to the right of $A$ in some sentential form.  The set of terminals of $a$ such that there exists a derivation of the form $S \to^* \alpha A a \beta$ for some $\alpha$ and $\beta$
\end{itemize}
\end{quote}

\textbf {Process to Compute $FIRST(X)$}
\begin{enumerate}
\item If $X$ is terminal, then $FIRST(X) = \{ X \}$
\item If $X\to \epsilon $ is a production, then add $\epsilon$ to $FIRST(X)$.
\item If $X$ is a non-terminal and $X\to Y_1 Y_2 ... Y_k$ is a production then $\forall Y_i \in X\to Y_i$ $FIRST(X) += FIRST(Y)$.  
\end{enumerate}

Process to compute $FOLLOW(A)$
\begin{enumerate}
\item Place end of string $\$ $ in $FOLLOW (S)$ where $S$ is the start symbol.
\item If there is a production $A\to \alpha B \beta$ where $\epsilon \notin FIRST(\beta)$ then make the following statement so: $FOLLOW(B) = FOLLOW(B) \cup FIRST(\beta) $
\item If there is a production $A\to \alpha B \beta$ where $\epsilon \in FIRST(\beta)$ hen make the following statement so: $FOLLOW(B) = FOLLOW(A) \cup FOLLOW(B) $
\end{enumerate}

\textbf {Selection Sets } (reference Cooke page IV-24)
\begin{enumerate}
\item The selection set of a non-$\epsilon$ right hand side option is the option's first set.
\item The selection set of a $\epsilon$ right hand side option is the option's follow set.
\end{enumerate}
``To complete the syntax analysis routines, one follows clearly defined rules for implementing the grammar specification''

Given Reserved Words and Symbols:
$\alpha_0 = program$
$\alpha_1 = array$
$\alpha_2 = integer$
$\alpha_3 = read$
$\alpha_4 = write $
$\alpha_5 = rdln $
$\alpha_6 = wrln $
$\alpha_7 = when$
$\alpha_8 = until$ 
$\alpha_9 = from$
$\alpha_{10} = ..$
$\alpha_{11} =  eof$
$\alpha_{12} = \{ $
$\alpha_{13} = \} $



 
\subsection {Given grammar:}
$P\to \alpha_0 D_1 \alpha_{12} S_L \alpha_{13}$
$D_1 \to I_L \alpha_{31} D $
$D_1 \to \epsilon $
$D \to \alpha _1 \alpha_{14} \alpha_{-2} D_2 \alpha_{15} \alpha_{16} D_1 $
$D \to \alpha_2 \alpha_{16} D_1$

$D_2 \to \alpha_{17} \alpha_{-2} D_2$
$D_2 \to \epsilon $
$I_L \to \alpha _{-1} I_L $
$I_{L_1} \to \alpha _{17} I_L  $
$I_{L_1} \to \epsilon $

Note that $I_D$ and $S_u$ requires left factoring 
$S_u \to \alpha_{-1} $
$S_u \to \alpha_{-2} $
$S_u \to \alpha_{-1} \alpha_{17} S_L$

$T\to I_D $
$T\to \alpha{11} $
$T\to \alpha_{-2} $


Left Factor Candidate
$S \to \alpha_{12} S_L \alpha_{13} S' $
$S \to \alpha_5 $
$S \to \alpha_6 $
$S \to \alpha_{18} I_L \alpha_{19} S''$

$S' \to \alpha_7 C$
$S' \to \alpha_8 C$
$S' \to \alpha_{-1} \alpha_{9} \alpha_{14} E \alpha_{10} E \alpha_{15}$

$S'' \to \alpha_{18} E_L \alpha_{19} \alpha_{20} $
$S'' \to \alpha_{3}$
$S'' \to \alpha_{4} $


$S_L \to S $
$S_L \to S \alpha_{16}$ 

Left Factor Candidate:
$C \to E_L \alpha _{16} E_L C'$
$C' \to \alpha _{21} C'$
$C' \to \alpha _{22} C'$
$C' \to \alpha _{23} C'$
$C' \to \alpha _{24} C'$
$C' \to  \alpha _{25} C'$
$C' \to \alpha _{26} C'$
$C'' \to \alpha_{16} C C'''$
$C''' \to  \alpha_{16} C \alpha_{31} C''$
$C''' \to  \alpha_{16} C \alpha_{32} C''$
$C''' \to  \alpha_{16} C \alpha_{33} C''$
$C''' \to  \alpha_{16} C \alpha_{34} C''$
$C'' \to \epsilon $

$E \to \alpha_{18} E_L \alpha_{19}  \alpha_{18} E_L \alpha_{19}  E'$
$E\to T$
$E'\to  \alpha_{27}$
$E'\to  \alpha_{28}$
$E'\to \alpha_{29}$
$E'\to  \alpha_{30}$
$E'\to  \alpha_{35}$

$E_L \to E E_L '$
$E_L \to E E_L $
$E_L \to \epsilon $

$C \to (E_L )(E_L) C'' $
$C' \to ; C C'''$
$C'' \to < C' | ...  | not C' | \epsilon$
$C'' \to \alpha _{21} C'$
$C'' \to \alpha _{22} C'$
$C'' \to \alpha _{23} C'$
$C'' \to \alpha _{24} C'$
$C'' \to  \alpha _{25} C'$
$C'' \to \alpha _{26} C'$
$C''' \to \alpha_{32} C' $
$C''' \alpha_{33} C' $
$C'''  \alpha_{34} C'$

$I_D	 \to  	\alpha_{-1} $
$ I_D \to \alpha_{14} S_u \alpha_{15} $

$S_u \to	\alpha_{-1} S'_u $
$S_u \to \alpha_{-2} S''_u$
$S'_u \to \alpha_{17} S_u $
$S''_u \to \alpha_{17} S_u $


\subsection {FIRST Sets:}
$P\to \alpha_0 D_1 \alpha_{12} S_L \alpha_{13}$ has the following:
$\alpha_0$

$D_1 \to I_L \alpha_{31} D $ $D_1 \to \epsilon $ 
has 
$\epsilon$, $\alpha_{-1}, \alpha_{31}$


$D \to \alpha _1 \alpha_{14} \alpha_{-2} D_2 \alpha_{15} \alpha_{16} D_1 $
$D \to \alpha_2 \alpha_{16} D_1$
has
$\alpha_{1}, \alpha_{2}$

$D_2 \to \alpha_{17} \alpha_{-2} D_2$
$D_2 \to \epsilon $
has
$\epsilon$, $\alpha_{17}$



$I_L \to \alpha _{-1} I_L $ has
$\alpha_{-1} $


$I_{L_1} \to \alpha _{17} I_L  $
$I_{L_1} \to \epsilon $
has
$\alpha_{17}$ $\epsilon$

$I_D	 \to  	\alpha_{-1} I'_D $
$\alpha_{-1}$
$ I'_D \to [ S_u ]  $
$ I'_D \to \epsilon $
has 
$\alpha_{-1}, \alpha_{-2}$, $\epsilon$

$T\to I_D $
$T\to \alpha_{11} $
$T\to \alpha_{-2} $
has:  
$\alpha_{11},\alpha_{-2}$, $\alpha_{-1}$

$S_u \to	\alpha_{-1} S'_u$
$S_u \to \alpha_{-2} S''_u$
$S'_u \to \alpha_{-1} \alpha_{17} S_u $
$S''_u \to \alpha_{-2} \alpha_{17} S_u $
has
$\alpha_{-1}, \alpha_{-2}$


$S \to \alpha_{12} S_L \alpha_{13} S' $
$S \to \alpha_5 $
$S \to \alpha_6 $
$S \to \alpha_{18} I_L \alpha_{19} S''$
has
$\alpha_{5}, \alpha_{6}, \alpha_{12}, \alpha_{18}$

$S' \to \alpha_7 C$
$S' \to \alpha_8 C$
$S' \to \alpha_{-1} \alpha_{9} \alpha_{14} E \alpha_{10} E \alpha_{15}$

has 
$\alpha_{7}, \alpha_{8}, \alpha_{-1}$


$S'' \to \alpha_{18} E_L \alpha_{19} \alpha_{20} $
$S'' \to \alpha_{3}$
$S'' \to \alpha_{4} $
has 
$\alpha_{18}, \alpha_{3}, \alpha_{4}$


$E \to \alpha_{18} E_L \alpha_{19}  \alpha_{18} E_L \alpha_{19}  E'$
$E\to T$
has:
$\alpha_{18}$, $\alpha_{11},\alpha_{-2}$, $\alpha_{-1}$

$E'\to  \alpha_{27}$
$E'\to  \alpha_{28}$
$E'\to \alpha_{29}$
$E'\to  \alpha_{30}$
$E'\to  \alpha_{35}$
has 
$\alpha_{27}, \alpha_{28}, \alpha_{29},  \alpha_{30}, \alpha_{35}$

$E_L \to E E'_L $
$E_L \to \epsilon $
has 
$\alpha_{18}$, $\alpha_{11},\alpha_{-2}$, $\alpha_{-1}$ $ \epsilon$
$E'_L \to \alpha_{17} E_L $
has 
$\alpha_{17}$

$C \to \alpha_{18} E_L \alpha_{19} \alpha_{18} E_L \alpha_{19} C'' $
has
$\alpha_{18}$


$C' \to \alpha_{16} C C'''$
has 
$\alpha_{16}$


%$C'' \to < C' | ...  | not C' | \epsilon$
$C'' \to \alpha _{21} C'$
$C'' \to \alpha _{22} C'$
$C'' \to \alpha _{23} C'$
$C'' \to \alpha _{24} C'$
$C'' \to  \alpha _{25} C'$
$C'' \to \alpha _{26} C'$
has 
$\alpha_{21}, \alpha_{22}, \alpha_{23}, \alpha_{24}, \alpha_{25}, \alpha_{26}$

$C''' \to \alpha_{32} C' $
$C''' \to \alpha_{33} C' $
$C'''  \to \alpha_{34} C'$
has
$\alpha_{32}, \alpha_{33}, \alpha_{34}$

$S_L \to S S'_L$
$S'_L \to \alpha_{16}$ 
$S'_L \to \epsilon$ 

$\alpha_{5}, \alpha_{6}, \alpha_{12}, \alpha_{18}$
 , $\alpha_{16}, \epsilon$


Follow Sets:
$P = \$ , $
$D_1 = \alpha_{12} \cup FOLLOW(D_1 )$
$D = \alpha_{12} \cup FOLLOW(D) $
$D_2 = \alpha_{15}$
$I_L = \alpha_{31}, \alpha_{19}, \alpha_{12}$
$I_{L_1} =  FOLLOW(I_{L_1})$
$S = \alpha_{13} \cup FOLLOW(S) $
$S' = \alpha_{13} $
$S'' = \alpha_{13} $
$S_u = \alpha_{15} \cup FOLLOW(S'_u) \cup FOLLOW(S''_u)$
$S'_u = FOLLOW(S_u)$
$S''_u = FOLLOW(S_u)$
$T = FOLLOW (E) $
$I'_D = FOLLOW(I_D) $
$I_D = FOLLOW(T) $

$E =\alpha_{10} , \alpha_{15}, \alpha_{16}, \alpha_{17}, \alpha_{19}$
$E = \cup FOLLOW()$
$E' = \cup FOLLOW(E)$
$E' = \alpha_{15}, \alpha_{16}, \alpha_{17}, \alpha_{19}$
$E_L = \alpha_{19}$
$E_L = \cup FOLLOW(E'_L)$
$E'_L = \alpha_{32}, \alpha_{33}, \alpha_{34} , \alpha_{13}$
$E'_L = \cup FOLLOW(E_L)$
$C = \alpha_{32}, \alpha_{33}, \alpha_{34}, \alpha_{13}$
$C' = \alpha_{32}, \alpha_{33}, \alpha_{34}, \alpha_{13}$
$C' = \cup FOLLOW(C'')$
$C'' = \cup FOLLOW(C)$
$C''' = \alpha_{32}, \alpha_{33}, \alpha_{34}, \alpha_{13}$
$C''' = \cup FOLLOW(C') $
$S_L = \alpha_{13}, \alpha_{19}, \alpha_{16} $
$S'_L = \alpha_{13} $
$S'_L = \cup FOLLOW(S_L)$

\subsection {Selection Sets}



\begin{table}[htdp]
\caption{Selection Sets for Given Grammar}
\begin{center}
\begin{tabular}{|c|c|c|c|}
\textbf{Production} & {First} & {Follow} & {Selection} \\
 $P\to \alpha_0 D_1 \alpha_{12} S_L \alpha_{13}$ & $\alpha_0$ & \$  & $\alpha_0$   \\
$D_1 \to I_L \alpha_{31} D $ & $\epsilon$, $\alpha_{-1}, \alpha_{31}$ & $\alpha_{12}$  & $\epsilon$, $\alpha_{-1}, \alpha_{31}$ \\
$D_1 \to \epsilon $ &  &  &  \\
$D \to \alpha _1 \alpha_{14} \alpha_{-2} D_2 \alpha_{15} \alpha_{16} D_1 $ & $\alpha_{1}, \alpha_{2}$  & $\alpha_{12}$ & $\alpha_{1}$ \\
$D \to \alpha_2 \alpha_{16} D_1$ &$\alpha_2$   &  &  $\alpha_2$ \\ 
$D_2 \to \alpha_{17} \alpha_{-2} D_2$ &  $\epsilon, \alpha_{17}$ & $\alpha_{15}$ &  $\alpha_{17}$\\ 
$D_2 \to \epsilon $ &  &  $\alpha_{15}$ &  \\ 
$I_L \to \alpha _{-1} I_{L_1} $ & $\alpha_{-1} $  &$ \alpha_{19} , \alpha_{31} $&  $ \alpha_{-1}$\\ 
$I_{L_1} \to \alpha _{17} I_L  $ &  $\alpha_{17}$ $\epsilon$ & $ \alpha_{19} , \alpha_{31}$  &  $\alpha_{17}$\\ 
$I_{L_1} \to \epsilon $ &  &  & $ \alpha_{19} , \alpha_{31}$  \\ 
$S_u \to	\alpha_{-1} S'_u $ & $\alpha_{-1}, \alpha_{-2}$ & $ \alpha_{15} $ & $\alpha_{-1}$  \\ 
$S_u \to \alpha_{-2} S''_u$ &  &  &   $\alpha_{-2}$\\ 
$S'_u \to \alpha_{17} S_u $ & $\alpha_{17}$  &  & $\alpha_{17}$  \\ 
$S''_u \to \alpha_{17} S_u $ & $\alpha_{17}$  &  &  $\alpha_{17}$ \\ 
$S \to \alpha_{12} S_L \alpha_{13} S' $ & $\alpha_{5}, \alpha_{6}, \alpha_{12}, \alpha_{18}$  & $\alpha_{13} $ &  $\alpha_{12}$ \\ 
$S \to \alpha_5 $ &$\alpha_{5} $  &$\alpha_{13} $  &  $\alpha_{5}$  \\ 
$S \to \alpha_6 $ &  $\alpha_{6} $ &$\alpha_{13} $  &   $\alpha_{6}$ \\ 
$S \to \alpha_{18} I_L \alpha_{19} S''$ &  $\alpha_{18}$ &  &   $\alpha_{18}$\\ 
$S' \to \alpha_7 C$ & $\alpha_{7}, \alpha_{8}, \alpha_{-1}$ & $\alpha_{13}$ &  $\alpha_{7}$ \\ 
$S' \to \alpha_8 C$ & $\alpha_{8}$  & $\alpha_{13} $ &   $\alpha_{8}$ \\ 
$S' \to \alpha_{-1} \alpha_{9} \alpha_{14} E \alpha_{10} E \alpha_{15}$ &  $\alpha_{-1}$ &$\alpha_{13} $  &  $\alpha_{-1}$ \\ 
$S'' \to \alpha_{18} E_L \alpha_{19} \alpha_{20} $ & $\alpha_{18}, \alpha_{3}, \alpha_{4}$ & $\alpha_{13}$ &  $\alpha_{18}$ \\ 
$S'' \to \alpha_{3}$ & $\alpha_{3}$ &  $\alpha_{13}  $ &   $\alpha_{3}$\\
$S'' \to \alpha_{4} $ & $\alpha_{4}$ &$\alpha_{13} $  &  $\alpha_{4}$\\

$T\to I_D $ &  $\alpha_{-1}, \alpha_{-2}$ & $\alpha_{10}, \alpha_{15}, \alpha_{17}, \alpha_{19}$ &  $\alpha_{-1}$ \\
$T\to \alpha_{11} $ & $\alpha_{11}$  & $\alpha_{10}, \alpha_{15}, \alpha_{17}, \alpha_{19}$ & $\alpha_{11}$ \\
$T\to \alpha_{-2} $ & $\alpha_{-2}$  & $\alpha_{10}, \alpha_{15}, \alpha_{17}, \alpha_{19}$ & $\alpha_{-2}$ \\

$I_D	 \to  	\alpha_{-1} I'_D $ & 
$\alpha_{-1}$  & $\alpha_{10}, \alpha_{15}, \alpha_{17}, \alpha_{19} $ & $\alpha_{-1}$ \\

$ I'_D \to \alpha_{14} S_u \alpha_{15}  $ &  $\alpha_{-1}, \alpha_{-2}$, $\epsilon$
&  $\alpha_{10}, \alpha_{15}, \alpha_{17}, \alpha_{19} $&  $\alpha_{-1}, \alpha_{-2}$, $\epsilon$\\

$ I'_D \to \epsilon $ &  &  $\alpha_{10}, \alpha_{15}, \alpha_{17}, \alpha_{19}$& $\alpha_{10}, \alpha_{15}, \alpha_{17}, \alpha_{19}$ \\
$E \to \alpha_{18} E_L \alpha_{19}  \alpha_{18} E_L \alpha_{19}  E'$ & $\alpha_{18}$ &  
$ \alpha_{15}, \alpha_{16}, \alpha_{17}, \alpha_{19}$
& $\alpha_{18}$ \\
$E\to T$ &  
 $\alpha_{11},\alpha_{-2}$, $\alpha_{-1}$ &  $ \alpha_{15}, \alpha_{16}, \alpha_{17}, \alpha_{19}$ &  $\alpha_{11},\alpha_{-2}$, $\alpha_{-1}$ \\

$E'\to  \alpha_{27}$  & $\alpha_{27}$ & $ \alpha_{15}, \alpha_{16}, \alpha_{17}, \alpha_{19}$   & $\alpha_{27}$ \\
$E'\to  \alpha_{28}$  & $\alpha_{28}$ &  &$\alpha_{28}$ \\
$E'\to \alpha_{29}$  & $\alpha_{29}$ &  & $\alpha_{29}$ \\
$E'\to  \alpha_{30}$  & $\alpha_{30}$ &  & $\alpha_{30}$\\
$E'\to  \alpha_{35}$ &
$\alpha_{27}, \alpha_{28}, \alpha_{29},  \alpha_{30}, \alpha_{35}$ & $ \alpha_{15}, \alpha_{16}, \alpha_{17}, \alpha_{19}$ & $\alpha_{27}$ \\

$E_L \to E E'_L $ & $\alpha_{18}  \alpha_{11},\alpha_{-2}$, $\alpha_{-1}$ &  & $\alpha_{18}  , \alpha_{11},\alpha_{-2}$, $\alpha_{-1}$ \\
$E_L \to \epsilon $ & 
$\alpha_{11},\alpha_{-2}$, $\alpha_{-1}$ $ \epsilon$ &
 $\alpha_{19}$ &
 $\alpha_{19}$  \\

$E'_L \to \alpha_{17} E_L $ &
$\alpha_{17}$ &  
$\alpha_{11}, \alpha_{-2}, \alpha_{-1}, \alpha_{19}$ & $\alpha_{17}$\\

$E'_L \to \epsilon$ &
 &  
 & $\alpha_{11}, \alpha_{-2}, \alpha_{-1}, \alpha_{19}$  \\

$C \to \alpha_{18} E_L \alpha_{19} \alpha_{18} E_L \alpha_{19} C'' $ & 
$\alpha_{18}$ &
$ \alpha_{32}, \alpha_{33}, \alpha_{34}, \alpha_{13}$ &   
$\alpha_{18}$ 

\end{tabular}
\end{center}
\label{default}
\end{table}%

\begin{table}[htdp]
\caption{Selection Sets for Given Grammar}
\begin{center}
\begin{tabular}{|c|c|c|c|}
\textbf{Production} & {First} & {Follow} & {Selection} \\
$C' \to \alpha_{16} C C'''$ &
$\alpha_{16}$ &
$ \alpha_{32}, \alpha_{33}, \alpha_{34}, \alpha_{13}$ &
$\alpha_{16}$  \\

$C'' \to \alpha _{21} C'$  & $\alpha_{21}$  &  & $\alpha_{21}$\\
$C'' \to \alpha _{22} C'$  &  &  &$\alpha_{22}$ \\
$C'' \to \alpha _{23} C'$  &  &  & $\alpha_{23}$\\
$C'' \to \alpha _{24} C'$  &  &  &$\alpha_{24}$ \\
$C'' \to  \alpha _{25} C'$  &  &  &$\alpha_{25}$ \\
$C'' \to \alpha _{26} C'$  &  
$\alpha_{21}, \alpha_{22}, \alpha_{23}, \alpha_{24}, \alpha_{25}, \alpha_{26}$ & $ \alpha_{32}, \alpha_{33}, \alpha_{34}, \alpha_{13}$ & $\alpha_{26}$ \\

$C''' \to \alpha_{32} C' $  & $\alpha_{32}$ &  & $\alpha_{32}$ \\
$C''' \to \alpha_{33} C' $ & $\alpha_{33}$ &  & $\alpha_{33}$ \\
$C'''  \to \alpha_{34} C'$ &
$\alpha_{32}, \alpha_{33}, \alpha_{34}$ & $ \alpha_{32}, \alpha_{33}, \alpha_{34}, \alpha_{13}$ &$\alpha_{34}$ \\

$S_L \to S S'_L$  & $\alpha_{5}, \alpha_{6}, \alpha_{12}, \alpha_{18}$  & $\alpha_{13} $  & $\alpha_{5}, \alpha_{6}, \alpha_{12}, \alpha_{18}$ \\
$S'_L \to \alpha_{16}$  &  $\alpha_{16}$ & $\alpha_{13} $ & $\alpha_{16}$ \\
$S'_L \to \epsilon$  & $\epsilon $ & $\alpha_{13} $ & $\alpha_{13} $
\end{tabular}
\end{center}
\label{default}
\end{table}%

\subsection {Syntax Analysis Completion}
\begin{quote}
$\forall$ LHS (or element of C) there is a boolean function:
$\forall$ RHS option of a given LHS there will be an if-statement block and if a given RHS option has multiple symbols,  there will be, in that block, a nested if for each symbol.  For a given RHS symbol there are three possible actions:
\begin{enumerate}
\item In the case of matching an element of the Vocabulary, $V$ such that  if $lex \in V:$ , consume the element (invoke the LA) and if the last item in the RHS option set the LHS symbol to true.  If you were supposed to match an element of $V$ and did not, set LHS to false and output an error message.  
\item In the case where the RHS is an element of the syntactic category, $C$: invoke C's function to see if it is true.  If true and if at the end of the RHS option, set the LHS symbol to true otherwise set the LHS to false.  
\item In the case of an epsilon production, check to see if $lex$ is an element of the selection set.  If it is, set the LHS to true.  Otherwise set LHS to false and put out an error message.  
\end{enumerate}

\end{quote}



 \end{document}


